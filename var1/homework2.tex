\documentclass[12pt]{article}
    
\usepackage[left=3cm,right=3cm,top=1cm,bottom=2cm]{geometry}
\usepackage{amsmath,amsthm}
\usepackage{amssymb}
\usepackage{lipsum}
\usepackage[T1,T2A]{fontenc}
\usepackage[utf8]{inputenc}
\usepackage[bulgarian]{babel}
\usepackage[normalem]{ulem}
    
\newcommand{\N}{\mathbb{N}}
\newcommand{\R}{\mathbb{R}}
    
\setlength{\parindent}{0mm}
        
\title{Вектори и матрици}
\author{Иво Стратев}
        
\begin{document}
\maketitle

\section*{Задача 1}

Напишете програма на любим за вас език за програмиране (ако нямате такъв може да пробвате Haskell),
която прочита две матрици записани във файлове matrix1.csv и matrix2.csv. Ако матриците не могат да бъдат събрани
отпечатва на екрана undefined. В противен случай отпечатва true или false в зависимост от това дали
резултата е симетрична матрица. \\

Една матрица $A$ наричаме симетрична ако е вярно равенството: $A^t = A$.

\section*{Задача 2}

Напишете програма на любим за вас език за програмиране (ако нямате такъв може да пробвате Haskell),
която прочита две матрици записани във файлове matrix1.csv и matrix2.csv. Ако матриците не могат да бъдат умножени
отпечатва на екрана undefined. В противен случай отпечатва true или false в зависимост от това дали
резултата е антисиметрична матрица. \\

Една матрица $A$ наричаме антисиметрична ако е вярно равенството: $A^t = -A$.

\section*{Задача 3}

Нека $n$ е фиксирано произволно естествено число. \\

Докажете, че за произволни два вектора $a$ и $b$ от $\R^n$ е изпълнено равенството: \\

$<a, \; b> = <b, \; a>$

\end{document}